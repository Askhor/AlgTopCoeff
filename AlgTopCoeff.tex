\documentclass{article}

\usepackage[utf8]{inputenc}
\usepackage{unicode-helper}
\usepackage{amsmath}
\usepackage{amssymb}
\usepackage{amsfonts}
\usepackage{microtype}
\usepackage[english]{babel}
\usepackage{amsthm}
\usepackage{xfrac}
\usepackage{mathtools}
\usepackage{braket}
\usepackage[pdftex]{hyperref}
\usepackage{cleveref}

\newcommand{\titlevar}{Algebraic Topology with different coefficients}
\newcommand{\authorvar}{Günthner}
\newcommand{\datevar}{WS24}
\title{\titlevar}
\author{\authorvar}
\date{\datevar}
\hypersetup{
	pdftitle=\titlevar,
	pdfauthor=\authorvar,
	pdfcreationdate=\datevar,
}
\setlength{\parindent}{0pt}

\DeclareMathOperator{\img}{img}

\begin{document}
	\maketitle

	This paper will be an execise in Algebraic Topology by developing singular homology with arbitrary coefficients.

	\medskip

	For this paper let $\mathbf{R}$ be a commutative ring and $T$ as topological space.

	\section{The homology of the point}

	\subsection{An Introduction of Homology}

	First let us define the $k$-Simplex:
	\begin{equation}
		Δ^k \coloneq \set{(t_i) ϵ ℝ^{k+1} : Σ_{i=1}^{k+1} t_i = 1 \text{ and } 0≤t_i≤1}
	\end{equation}

	Then we define $C_k(T)$ as the free module over $\mathbf{R}$ with basis:
	\begin{equation}
		\set{Δ^k → T}
	\end{equation}

	Now we define $d_k: C_k(T) → C_{k-1}(T)$ to make the $C_k{T}$ a chain complex, meaning that $d²=0$:
	\begin{equation}
		d_k(φ) = Σ_{i=0}^{k} (-1)^i d_k^i(φ)
	\end{equation}

	The $d_k^i$ is defined as follows:
	\begin{equation}
		d_k^i(φ)(t_j) = φ(t_1, t_2, \dots, t_{i-1}, 0, t_{i+1} , \dots, t_{k-2}, t_{k-1})	
	\end{equation}

	\medskip

	Homology can now be defined as:
	\begin{equation}
		H_k(T) \coloneq \sfrac{\ker(d_k)}{\img(d_{k+1})}
	\end{equation}

	\subsection{The calculation}

	To calculate $H_k(*)$, we will need to figure out when $d_k$ is the identity and when it is the zero-map. First it will be good to understand the underlying chaincomplex:
	\begin{equation}
		C_k(*) = \set{r · (Δ^k → * \text{ constant}) : rϵ\mathbf R}
	\end{equation}
	In other words $\dim(C_k(*)) = 1$.

	\medskip

	Now if $k$ is odd there is an even number of boundries and all the terms in the alternating sum of $d_k$ cancel each other out and $d_k = 0$.

	\medskip

	If—however—$k$ is even it is not the null-map and (as $\dim(C_k(*))=1$) $d_k$ is an isomorphism.
	
	\medskip

	This can be used to compute the kernel and the image of $d_k$:
	\begin{equation}
		\ker(d_k) =
		\begin{cases}
			\mathbf{R} &	\text{if } k \text{ odd}\\
			0 	   &	\text{if } k \text{ even}\\
		\end{cases}
	\end{equation}
	\begin{equation}
		\img(d_k) =
		\begin{cases}
			0		& \text{if } k \text{ odd}\\
			\mathbf{R}	& \text{if } k \text{ even}\\
		\end{cases}
	\end{equation}
	
	\medskip

	This yields for $k>0$:
	\begin{equation}
		H_k(*) = 0
	\end{equation}

	\medskip

	For $k=0$:
	\begin{equation}
		H_0(*) = \sfrac{\mathbf{R}}{0} = \mathbf{R}
	\end{equation}
\end{document}
